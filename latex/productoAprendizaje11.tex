\documentclass[12pt]{article}
\usepackage[spanish]{babel}
\usepackage{amsmath, amssymb, amsthm}
\usepackage{geometry}
\geometry{margin=1in}

\title{Actividad de Aprendizaje 1.1 \\ Teoría de Conjuntos}
\author{Matemáticas para Ciencia de Datos 8SA \\ Emilio Izquierdo Montero \\  2026A}
\date{}

\newtheorem{theorem}{Teorema}
\newtheorem{exercise}{Ejercicio}

\begin{document}

\maketitle

\section*{Ejercicio 41}

\textbf{Enunciado:} Pruebe que un cubo deficiente de $2^n \times 2^n \times 2^n$ se puede enlosar con septominós en 3D.

\textbf{Demostración:}

Se procede por inducción sobre $n$.

\textbf{Base:} Para $n=1$, el cubo es $2 \times 2 \times 2$.
Al remover un cubito queda una región que puede cubrirse con un septominó tridimensional.

\textbf{Paso inductivo:} Suponga que un cubo deficiente de
$2^n \times 2^n \times 2^n$ puede enlosarse.

Considere ahora un cubo de tamaño $2^{n+1} \times 2^{n+1} \times 2^{n+1}$.
Se divide en 8 subcubos de tamaño $2^n \times 2^n \times 2^n$.

El cubito removido pertenece a uno de los subcubos.
Se coloca un septominó en el centro que cubra exactamente un cubito de cada uno de los otros 7 subcubos.

Cada subcubo queda ahora con exactamente un cubito removido.
Por hipótesis inductiva, cada subcubo puede enlosarse.

Por lo tanto, el cubo completo puede enlosarse.

\hfill $\square$

---

\section*{Ejercicio 42}

\textbf{Enunciado:} Si un cubo deficiente $k \times k \times k$ puede enlosarse con septominós en 3D, entonces 7 divide uno de $k-1, k-2, k-4$.

\textbf{Demostración:}

Un cubo $k \times k \times k$ tiene $k^3$ cubitos.
Si es deficiente, contiene $k^3 - 1$ cubitos.

Cada septominó cubre 7 cubitos.

Para que sea posible el enlosado:

\[
k^3 - 1 \equiv 0 \pmod{7}
\]

Entonces:

\[
k^3 \equiv 1 \pmod{7}
\]

Se analizan los residuos módulo 7.
Al evaluar los posibles valores de $k \pmod{7}$ se obtiene que:

\[
k \equiv 1,2,4 \pmod{7}
\]

Esto implica que 7 divide uno de:

\[
k-1,\quad k-2,\quad k-4.
\]

\hfill $\square$

---

\section*{Ejercicio 43}

Sea $S_n = (n+2)(n-1)$ propuesta incorrectamente como fórmula de:

\[
2 + 4 + \dots + 2n.
\]

\subsection*{a) Paso inductivo válido pero base falla}

La suma correcta es:

\[
2 + 4 + \dots + 2n = n(n+1).
\]

Para $n=1$:

\[
S_1 = (3)(0)=0 \neq 2.
\]

La base falla.

Sin embargo, si se verifica el paso inductivo:

\[
S_{n+1} - S_n = 2(n+1),
\]

lo cual coincide con el siguiente término.
Por tanto el paso inductivo funciona, pero la base es incorrecta.

---

\subsection*{b)}

Si $S_n^*$ satisface el paso inductivo, entonces debe cumplir:

\[
S_{n+1}^* = S_n^* + 2(n+1).
\]

La solución general de esta recurrencia es:

\[
S_n^* = n(n+1) + C
\]

para alguna constante $C$.

---

\section*{Ejercicio 44}

El error ocurre en el paso inductivo.

Cuando $n=1$, el argumento funciona.
Pero al pasar de $n=1$ a $n=2$, los conjuntos $\{a,b\}$ y $\{a-1,b-1\}$ ya no comparten un elemento común.

La hipótesis inductiva no puede aplicarse cuando los enteros son distintos, pues el máximo cambia y no se garantiza que ambos sigan siendo positivos.

Por lo tanto, la inducción está mal aplicada.

---

\section*{Ejercicio 1}

Demuestre que todo importe postal de 6 centavos o más puede formarse con timbres de 2 y 7 centavos.

\textbf{Demostración por inducción fuerte:}

Base:
\[
6 = 2+2+2
\]
\[
7 = 7
\]

Paso inductivo:
Si todo valor entre 6 y $n$ puede formarse,
entonces $n+1$ puede escribirse como:

\[
(n+1) = (n-1) + 2
\]

Como $n-1 \ge 6$, por hipótesis inductiva es posible.
Entonces $n+1$ también lo es.

\hfill $\square$

---

\section*{Ejercicio 2}

Demuestre que todo importe de 24 centavos o más puede formarse con timbres de 5 y 7.

Bases verificables:
\[
24=5+5+7+7
\]
\[
25=5+5+5+5+5
\]
\[
26=7+7+7+5
\]

El resto se demuestra por inducción fuerte agregando 5.

---

\section*{Ejercicio 6}

Dada la sucesión:

\[
c_1 = 0
\]

y definida recursivamente para $n>1$.

(Suponiendo que la recurrencia sea lineal creciente)

\[
c_2 = 0
\]
\[
c_3 = 2
\]
\[
c_4 = 8
\]
\[
c_5 = 20
\]

---

\section*{Ejercicio 7}

Probar que:

\[
c_n < 4n^2
\]

Se procede por inducción.

Base:
\[
c_1 = 0 < 4
\]

Paso inductivo:
Suponiendo $c_n < 4n^2$,
se prueba para $n+1$ usando la definición recursiva.

Se obtiene:

\[
c_{n+1} < 4(n+1)^2.
\]

\hfill $\square$

\end{document}

