\documentclass[12pt]{article}
\usepackage[spanish]{babel}
\usepackage{amsmath, amssymb}
\usepackage{graphicx}
\usepackage{geometry}
\usepackage{float}
\usepackage{hyperref}
\geometry{a4paper, margin=2.5cm}

\title{Actividad de Aprendizaje 1.2 \\ Investigación de Funciones \\ emilio izquierdo montero}
\author{Matemáticas para Ciencia de Datos \\ 8SA 2026A}
\date{}

\begin{document}

\maketitle
\tableofcontents
\newpage

\section{Introducción}

Las funciones matemáticas permiten modelar fenómenos reales estableciendo una relación entre variables. En Ciencia de Datos, las funciones son fundamentales para describir comportamientos, realizar predicciones y analizar información. En este documento se presentan cinco aplicaciones prácticas de distintos tipos de funciones.

\section{Función Lineal: Modelado de Costos}

Una función lineal tiene la forma:

\[
f(x) = mx + b
\]

donde:
\begin{itemize}
    \item $m$ es la pendiente
    \item $b$ es la intersección con el eje $y$
\end{itemize}

\subsection*{Aplicación}

Supongamos que una empresa cobra \$50 pesos por envío más \$10 pesos por cada producto adicional. El costo total se modela como:

\[
C(x) = 10x + 50
\]

donde $x$ es la cantidad de productos.

Esta función permite predecir ingresos y planificar estrategias comerciales.

\section{Función Cuadrática: Movimiento y Optimización}

Una función cuadrática tiene la forma:

\[
f(x) = ax^2 + bx + c
\]

\subsection*{Aplicación}

En física, la altura de un objeto lanzado verticalmente puede modelarse como:

\[
h(t) = -4.9t^2 + v_0 t + h_0
\]

Esta función permite:
\begin{itemize}
    \item Calcular la altura máxima.
    \item Determinar el tiempo de impacto.
\end{itemize}

En Ciencia de Datos también se usa en problemas de optimización.

\section{Función Exponencial: Crecimiento Poblacional y Datos}

Una función exponencial tiene la forma:

\[
f(x) = a e^{bx}
\]

\subsection*{Aplicación}

El crecimiento de usuarios en una red social puede modelarse como:

\[
U(t) = U_0 e^{kt}
\]

donde:
\begin{itemize}
    \item $U_0$ es la cantidad inicial de usuarios.
    \item $k$ es la tasa de crecimiento.
\end{itemize}

Se utiliza para:
\begin{itemize}
    \item Modelar crecimiento acelerado.
    \item Analizar propagación de información.
\end{itemize}

\section{Función Logarítmica: Escalas y Análisis de Datos}

Una función logarítmica tiene la forma:

\[
f(x) = \log(x)
\]

\subsection*{Aplicación}

El nivel de sonido en decibeles se calcula mediante:

\[
L = 10 \log_{10}\left(\frac{I}{I_0}\right)
\]

También se usa en Ciencia de Datos para:
\begin{itemize}
    \item Normalizar datos.
    \item Reducir sesgos en distribuciones muy dispersas.
\end{itemize}

\section{Función Sigmoide: Clasificación en Machine Learning}

La función sigmoide se define como:

\[
\sigma(x) = \frac{1}{1 + e^{-x}}
\]

\subsection*{Aplicación}

Se usa en regresión logística para clasificar datos en categorías binarias (por ejemplo: aprobado/reprobado).

Convierte cualquier valor real en un valor entre 0 y 1, interpretado como probabilidad.

\section{Conclusión}

Las funciones matemáticas son herramientas fundamentales en Ciencia de Datos. Permiten modelar fenómenos económicos, físicos y tecnológicos. Comprender sus propiedades facilita el análisis, predicción y toma de decisiones basada en datos.

\end{document}

